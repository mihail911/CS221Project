\documentclass[12pt]{article}
\usepackage{fullpage,enumitem,amsmath,amssymb,graphicx}
\newcommand{\SUMN}[3]{\displaystyle\sum\limits_{n={#1}}^{{#2}} {#3}}
\newcommand{\SUMK}[3]{\displaystyle\sum\limits_{k={#1}}^{{#2}} {#3}}
\newcommand{\SUMX}[3]{\displaystyle\sum\limits_{x={#1}}^{{#2}} {#3}}
\newcommand{\SUMI}[3]{\displaystyle\sum\limits_{i={#1}}^{{#2}} {#3}}
\newcommand{\SUMJ}[3]{\displaystyle\sum\limits_{j={#1}}^{{#2}} {#3}}

\linespread{1.5}
\begin{document}

\begin{center}
{\Large OF COURSE! A Bayesian Inference Approach to More Dynamic Course Seach}
\begin{center}
{\normalsize CS221 Final Project by Michael Dickens and Mihail Eric}\\
\today 
\end{center}

\end{center}



\section*{TODOs}

%
Metric to employ: \\
If no classes returned, search result of 0.\\

Only top 5 course searches considered in evaluation.\\

Things to consider:\\
50 points if course desired in first hit, then 30, 25, 20, etc.\\
1) if course in same sequence as searched for course (+10,+8,...)\\
2) if course in same department (+5,+4,+3,+2,+1)\\
3) if course is same course in a different department (+10)\\
4) if course has same instructor (+10,+8,+6,+4,+2)\\
5) if course descriptions are sufficiently similar, judged subjectively (+5,+4,...)\\
% 5) number of nouns that course descriptions have in common (with expected course)\\
------------------------------------------------------------------\\
If specific course being searched, then we know that the user has a particular course in mind for their search.

Data (25 searches across different 5 different domains (5 searches per domain))\\
1) specific course code\\
2) instructor name\\
3) course title searchs (including segments of course titles)\\
4) department code\\
5) more complex queries ('courses taught by', ``cs109 and cs154'')



\section*{Introduction}
ExploreCourses is a search engine used regularly by the Stanford
University community for browsing and finding courses. At the present
moment, ExploreCourses can support basic query searches, including
somewhat accurate retrieval given a course code and the exact title of
a course. Almost any query search that does not fall into one of these
two classes of searches will either return unrelated class results or
more often no results at all. For this project, we sought to develop
an improved course searching program that would be more robust in that
it could support more diverse user input queries and would also be
more dynamic in that the courses that were returned were more related
given a universal metric that we define for assessing relatedness. The
metric will be explained in a later section.

To achieve improved robustness, we implemented some basic natural
language processing schemes for extracting useful and relevant
information from a user input. The information that we were
specifically looking for included course titles, course codes,
department codes, and instructor names. To find more related courses,
we extracted a variety of features that we considered relevant from
all the data we could attain about a course and then we created a
course-relatedness ``graph'' that assigns a relatedness score to each
pair of classes, given their extracted features. To compute the
relatedness score, we utilized a Bayesian inference scheme whereby we
calculated the probabilities that two courses are related given that
they have a pair of features in common. The Bayesian approach and the
features used will be explained in later sections.

\section*{Feature Extraction}
In order to determine an accurate label for relatedness
between two courses, we had to extract a useful collection of
features from the data for each course. This required some
experimentation in order to find a good balance: too many features and
the algorithm runs slowly; too few, and we cannot perform useful
inference.

We used the following initial set of features: 

\begin{itemize}
\item words in the title
\item words in the description
\item course code name
\item course code ones digit
\item course code tens digit
\item instructors
\item minimum units
\item maximum units
\end{itemize}

Later, we took each of the course code features and combined them with
each title, instructor, and description feature to create a set of
binary features. This roughly quadruples the number of features in the
set. We also tried using word bigrams in the title and description,
but this did not add substantial benefit.

\section*{A Bayesian Approach to Course-Relatedness}
**NOTE: THIS IS JUST A COPY-PASTE OF MICHAEL's NOTES!!!\\

First I implemented naiveWeights(), which worked sort of well. Then I implemented bayesWeights(). Finding P(R) and P(F) wasn't too hard, but how to find P(F|R)?

Also, at a certain point I realide I also need to update on a feature not being in the second set as well as a feature being in the second sent--these both provide Bayesian evidence.

Right now for P(F|R) and P(~F|R) I just threw in some constants. This seems to work okay.

Then there's the question of how to combine each conditional
probability. First I tried using combineProbs(), but that doesn't seem
to work too well. It works a lot better to simply take the sum of all
the probabilities. Why?

**** Actual description ****

To determine course relatedness, we began simply by taking the total
number of matching features between two courses. This approach proved
too coarse: it matched courses with many common features that weren't
actually all that related to each other.

A sudden insight came when we realized we could do much better by
taking a Bayesian appraoch. Instead of simply counting the number of
features in common, interpret each feature in common as Bayesian
evidence that the two courses are related and perform a Bayesian
probability update. Similarly, if a feature does not occur in common
between two courses, consider this evidence that they are not related.

\section*{Query Parsing}
In order to satisfy a user's input queries, we need to extract useful information from a given query. Using the Python Natural Language Processing Toolkit, we employed the following natural language processing scheme: tokenize query, tag with parts of speech, chunk appropriately using regular expression grammars. Once a query is chunked, useful information can be derived through analysis of the corresponding parse tree. Using this scheme, we are able to support user query searches consisting of more complex phrases 

We search for instructors using a simple grammar that searches a string for sequences of proper nouns. We support course code and department code searches by tokenizing a query into unigram and bigram tokens and checking a set of course/department codes for matches. We also support title searches by searching for the input string in a set of all course titles.\\
Talk about use of NLTK POS-Tagger, chunker, regex grammar.

\section*{Data}
	 We used the ExploreCourses Java API to acquire information related to the 11,613 courses listed for enrollment for the 2013-2014 academic year. We populated a SQL datebase, using the sqlite3 Python library, with the following information for each course: course title, course code, instructors teaching the course, minimum units of credit received for taking the class, maximum units units of credit received for taking the class, and course description. \\

	 However, for the purposes of the assessment, we used a database of a reduced subset of approximately 170 random classes taken from the CS and MATH departments. We had to utilize a reduced database because it was too time-consuming to create a comprehensive relatedness graph for all 11,613 courses.

\section*{Metric for Assessment}
	We developed a simple point-based metric to objectively
        determine the quality of our course searcher as compared to ExploreCourses. To evaluate the performance of our searcher, we generated a random subset of 25 courses taken from our reduced database. We subdivided these 25 courses into 5 domains of query searches with 5 courses per domain: searching a specific course code such as `CS109', searching for an instructor name such as `Mehran Sahami', searching for a course title or subset of such as 
3) course title searchs (including segments of course titles)\\
4) department code\\
5)more complex queries ('courses taught by', ``cs109 and cs154'')
	The robustness of our searcher as compared to ExploreCourses using our metric 
\section*{Results}
	
\section*{Further Work}
**Support searches based on user's history using more standard supervised machine learning classifiers.\\
	**More complex query searches\\
	**Spell-correction (edit distance)\\
	**Play around with other schemas for determining relatedness coefficient\\
	**Incorporate into the actual site.\\

\section*{References}


\end{document}
