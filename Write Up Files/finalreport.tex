\documentclass[12pt]{article}
\usepackage{fullpage,enumitem,amsmath,amssymb,graphicx,indentfirst}
\newcommand{\SUMN}[3]{\displaystyle\sum\limits_{n={#1}}^{{#2}} {#3}}
\newcommand{\SUMK}[3]{\displaystyle\sum\limits_{k={#1}}^{{#2}} {#3}}
\newcommand{\SUMX}[3]{\displaystyle\sum\limits_{x={#1}}^{{#2}} {#3}}
\newcommand{\SUMI}[3]{\displaystyle\sum\limits_{i={#1}}^{{#2}} {#3}}
\newcommand{\SUMJ}[3]{\displaystyle\sum\limits_{j={#1}}^{{#2}} {#3}}

\newcommand{\todo}[1]{{\Large \bf ***TODO: #1***}}

\linespread{1.5}
\setlength\parindent{24pt}
\begin{document}

\begin{center}
{\Large OF COURSE! Using Bayesian Inference to Build a More Dynamic
  Course Search}

{\normalsize CS221 Final Project by Michael Dickens and Mihail Eric}\\
\today 
\end{center}

\setlist[enumerate]{itemsep=0mm}
\setlist[itemize]{itemsep=0mm}

\section*{Introduction}
ExploreCourses is a search engine used regularly by the Stanford
University community for browsing and finding courses. At the present
moment, ExploreCourses can support basic query searches, including
somewhat accurate retrieval given a course code and the exact title of
a course. Almost any query search that does not fall into one of these
two classes of searches will either return unrelated class results or
more often no results at all. For this project, we sought to develop
an improved course searching program that would be more robust in that
it could support more diverse user input queries and would also be
more dynamic in that the courses returned would be more related
given a universal metric that we define for assessing relatedness. The
metric is explained in a later section.

To achieve improved robustness, we implemented some basic natural
language processing schemes for extracting useful and relevant
information from a user input. The information that we were
specifically looking for included course titles, course codes,
department codes, and instructor names. To find more related courses,
we extracted a variety of features that we considered relevant from
all the data we could attain about a course and then we created a
course-relatedness ``graph'' that assigns a relatedness score to each
pair of classes, given their extracted features. To compute the
relatedness score, we utilized a Bayesian inference scheme whereby we
calculated the probabilities that two courses are related given that
they have a pair of features in common. The Bayesian approach and the
features used will be explained in later sections.

\section*{Feature Extraction}
In order to determine an accurate label for relatedness
between two courses, we had to extract a useful collection of
features from the data for each course. This required some
experimentation in order to find a good balance: too many features and
the algorithm runs slowly; too few, and we cannot perform useful
inference.

We used the following initial set of features: 

\begin{itemize}
\item stemmed words in the title
\item words in the description
\item course code name
\item course code ones digit
\item course code tens digit
\item instructors
\item minimum units
\item maximum units
\end{itemize}  

Later, we took each of the course code features and combined them with
each title, instructor, and description feature to create a set of
binary features. This roughly quadruples the total number of features
in the set. We also tried using word bigrams in the title and
description, but this did not add substantial benefit.

\section*{A Bayesian Approach to Course-Relatedness}

To determine course relatedness, we began simply by taking the total
number of matching features between two courses. This approach proved
too coarse: it matched courses with many common features that weren't
actually all that related to each other.

A sudden insight came when we realized we could do much better by
taking a Bayesian appraoch. Instead of simply counting the number of
features in common, interpret each feature in common as Bayesian
evidence that the two courses are related and perform a Bayesian
probability update. Similarly, if a feature does not occur in common
between two courses, consider this evidence that they are not related.

Then, instead of considering each feature as equally strong evidence,
weigh a feature against the prior probability of that feature
occurring. We figure out the prior probability of a feature by
counting how frequently it occurs in the database.

Thus, to find the probability that two courses are related, we update
a prior probability estimate with the evidence given by each feature
found in the two courses: \\

$P(related | feature) = P(related) \dfrac{P(feature | related)}{P(feature)}$ \\

We had some difficulty in combining probabilities. We attempted to use
the naive Bayes assumption to compute the combined probability (Graham), but
this gave us unworkably-small probabilities. We updated the model to
combine each of the feature probabiliteis by simply taking their
sum. Although this operation is not meaningful in a probabilistic
sense, it works as a useful heuristic for producing accurate results.

\section*{Query Parsing}
In order to satisfy a user's input queries, we need to extract useful
information from a given query. Using the Python Natural Language
Processing Toolkit, we employed the following natural language
processing scheme: tokenize query, tag with parts of speech, chunk
appropriately using regular expression grammars. Once a query is
chunked, useful information can be derived through analysis of the
corresponding parse tree. Using this scheme, we are able to support
user query searches consisting of more complex phrases including ones of the form 
`courses taught by Kannan Soundararajan.'

As an example of how our program processes a query, consider the
example from above: `courses taught by Kannan Soundararajan.' We first
tokenize the query and assign part-of-speech tags to each token. Note
that we used the built-in Python NLTK part-of-speech (POS) tagger
which is trained on the Penn Treebank corpus. This gives us the
following output: \vspace{0.15in}

\emph{\small{[(`courses', `NNS'), (`taught', `VBD'), (`by', `IN'), (`Kannan', `NNP'), (`Soundararajan', `NNP')]}}	 \vspace{0.15in}

\noindent where the second entry in each tuple represents the corresponding
part-of-speech that the token has been identified as. Now, using a
regular expression grammar of the form ``PNOUN:\{ $<$ NNP$>$*\}'', we
can chunk the text to get the following parse tree in flattened text
form: \vspace{0.15in}

\emph{\small{Tree(`S', [(`courses', `NNS'), (`taught', `VBD'), (`by', `IN'), Tree(`PNOUN', [(`Kannan', `NNP'), (`Soundararajan', `NNP')])])
}} \vspace{0.15in}

\noindent Here the grammar searches for strings of proper nouns, operating under the assumption that most strings of proper nouns in a user input will be associated with the name of an instructor. Thus, given this parse tree, we can deduce that the user wants to search for courses taught by `Kannan Soundararajan.'


We support course code and department code searches by tokenizing a
query into unigram and bigram tokens and linearly searching for
matches against a set of course/department codes. This scheme allows
us to find multiple course codes in a user query. We also support
title searches by searching for the input string in a set of all
available course titles. In order to satisfy more complex query
searches, we have a few parsing grammars in place for extracting
information. A few of them include:
\begin{enumerate}
	\item \{$<$NNP$>$*\}. For identifying strings of proper nouns.

	\item \{$<$VBD$>$ $<$IN$>$ $<$NNP$>$*\}. For identifying phrases of the form: verb, proposition, proper noun.
\end{enumerate}

As more grammars are added, the complexity of searches supported will increase.


\section*{Data}
We used the ExploreCourses Java API to acquire information related to
the 11,613 courses listed for enrollment for the 2013–2014 academic
year. We populated a SQL datebase, using the sqlite3 Python library,
with the following information for each course: course title, course
code, instructors teaching the course, minimum units of credit
received for taking the class, maximum units units of credit received
for taking the class, and course description.

However, for the purposes of the assessment, we used a database of a
reduced subset of approximately 170 random classes taken from the CS
and MATH departments. We had to utilize a reduced database because it
was too computationally time-consuming to create a comprehensive relatedness graph for
all 11,613 courses.

We use the query parser to find an exact match for the query. Then we
use the course relatedness graph to find a set of the nearest
courses. A query returns this set of courses in order of relatedness.

For example, if the user inputs `CS221', we identify the course with
the code `CS221'. Then we find the courses most similar to CS221 and
return those.

This approach has certain limitations: for example, a course that has
CS221 as a prerequisite might be more relevant to a search for `CS221'
but not be considered sufficiently closely related to CS221. In
practice, this isn't much of a problem. Our current approach seems to
work relatively well, but we may refine it in the future.

\section*{Metric for Assessment}
We developed a simple point-based metric to objectively determine the
quality of our course searcher as compared to ExploreCourses. To
evaluate the performance of our searcher, we generated a random subset
of 21 courses taken from our reduced database. We subdivided these 21
courses into the following types of searches: 

\begin{enumerate}
\item Specific course code such as CS109
\item Instructor name such as `Mehran Sahami'
\item Course title or subset of course title such as `Probability for Computer Scientists'
\item `courses taught by [instructor]'
\item `[course 1] and [course 2]'
\end{enumerate}

We used the following point system to evaluate the accuracy of queries: 

For each of the following, give the full points for the first hit, 3/5
of the points for the second hit, and 2/5 the points for the third
hit. We only consider the top 3 hits.

\begin{itemize}
\item 100 points if exact match OR
\item 10 points if same sequence as searched-for course (e.g. Math 51,
  52, 53)
\item 5 points if in the same department
\item 10 points if has same instructor
\end{itemize}

Searches for instructors are judged simply by whether the result is
taught by the instructor. An exact match is worth 100 points.

These numbers are somewhat arbitrary. To get a better idea of the
reliability of our search algorithm, we should collect data on user
satisfaction with real-world queries. Unfortunately, such a metric is
not feasible at this time, so for now we will stick with the
point-based system described above.

\section*{Results}

We compared ExploreCourses and our search using the methods decribed
above. ExploreCourses successfully found courses for exact course code
searches (e.g. `Math120'), occasionally found courses for title
searches (e.g. `introduction to computing principles'), and failed for
all other types of searches.

Our improved course search performed roughly as well as ExploreCourses
on these two types of searches. In addition, it performed much better
for other queries. It had 100 percent success for `courses taught by
[instructor]' or just `[instructor]' and performed reasonably well on
the more complex queries. There exists some room for improvement in
queries for specific course codes (arguably the most important type of
query); we would like to achieve 100 percent success for course code
queries. Our problem here is fairly simple: if the course code has a
space in it (e.g. `Math 53' rather than `Math53'), our search
fails. We should be able to correct this in a future iteration of the
program.

Its greatest weakness arose when searching for course titles: it only
successfully found the intended course about a third of the time. It
perfomed better in some preliminary tests that we did, and we believe
that we may have simply modified the program in some way that created
a bug. This should not be difficult to fix.

In total, ExploreCourses scored 750 points and our search scored 1290
points. Here is the breakdown by section: 

\begin{tabular}{l r r r}
type & ExploreCourses & our search & max possible \\
code & 400 & 375 & 500\\
instructor & 0 & 100 & 100 \\
title & 300 & 315 & 1100 \\
taught by X & 0 & 300 & 300 \\
X and Y & 0 & 200 & 300 \\
\end{tabular}
	
\section*{Further Work}

In the future, we can make two classes of improvements to our program:
augmentations to the artificial intelligence schemes we employ to make
our program `smarter'; and additions to improve the general robustness
of the program's features.

First, let us discuss some of the improvements to the artificial
intelligence infrastructure we can implement. At the moment, our query
parsing uses the built-in NLTK POS-tagger which has made specific
mistakes on some queries in the past such as occasionally labeling
`CS109' as a proper noun. We may want to generate our own corpus of
expected user queries along with part-of-speech tagging, so that we
can more appropriately train the POS-tagger that will be used for our
program. We will also want to support a more diverse collection of
chunking grammars that will allow for a greater complexity of query
searches. Another thing we've also brainstormed is the possibility of
supporting searches based on a user's previous history. Here we could
employ some standard supervised machine learning classifying
algorithms. We could use a similar Bayesian inference scheme whereby
we find the probability that a user will want to find a certain course
given that she has searched a number of other classes in the past.

As far as improving the overall robustness of the program, the code is
definitely not implemented as efficiently as it could be at this
time. We often do complete linear searches through huge databases to
find matches; we could greatly improve this by managing our data
within a more intelligent data structure. The computational slowness
of the code is also the reason we had to develop a reduced databased
for testing course-relatedness. In the future we would like to create
a relatedness graph for the entire collection of 11,000 courses
offered. 

We may also implement a spell-correction algorithm,
for fixing a query if it does not directly match a name or code
available in a database. Something as basic as a memoized edit
distance algorithm should work very well for our purposes. Given all
these places for improvement, our searcher already outperforms
ExploreCourses, and we would like to get our code live at some point
in the future. We are already in communication with the Office of the
University Registrar and they have expressed some excitement about our
improved course search.



\section*{References}

\indent Bird, Steven, et al. (2009). Natural Language Toolkit. Retrieved from http://nltk.org/\\

Gabrilovich, E., \& Markovitch, S. (2009). Wikipedia-based semantic interpretation for natural language processing. Journal of Artificial Intelligence Research, 34(2), 443.\\

Graham, Paul, ``Probability.'' Retrieved from http://www.paulgraham.com/naivebayes.html.


\end{document}
